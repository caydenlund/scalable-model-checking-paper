%%%%%%%%%%%%%%%%%%%%%%%%%%%%%%%%%%%%%%%%%%%%%%%%%%%%%%%%%%%%%%%%%%%%%%%%%%%%%%%%
%  File:   results-and-discussion.tex                                          %
%  Author: Cayden Lund (cayden.lund@utah.edu)                                  %
%          Student, University of Utah                                         %
%                                                                              %
%  Brief:  Contains the section on results and discussion.                     %
%%%%%%%%%%%%%%%%%%%%%%%%%%%%%%%%%%%%%%%%%%%%%%%%%%%%%%%%%%%%%%%%%%%%%%%%%%%%%%%%

\subsection{Comparing the Performance and Effectiveness of \texttt{rumur} and \texttt{romp}}\label{subsec:comparing-the-performance-and-effectiveness-of-rumur-and-romp}

For the Tic-Tac-Toe strategy model, both \texttt{rumur} and \texttt{romp}
successfully identified the flaw in the strategy that allowed the opponent to
win under a certain sequence of moves.
The exhaustive enumeration approach employed by \texttt{rumur} provided a
definitive answer, confirming the presence of the flaw.
On the other hand, \texttt{romp} utilized its parallel random walk strategy
to efficiently explore the state space and identify the same flaw.
Both tools proved effective in detecting and localizing the bug, enabling the
correction of the flawed strategy.

In the case of the leader election protocol model, the size of the state
space proved too large for \texttt{rumur} to perform a comprehensive analysis.
As a result, we could not obtain conclusive results from \texttt{rumur}
regarding the correctness of the model or the identification of flaws.
\texttt{romp}, however, with its ability to handle large state spaces through
parallel random walks, successfully identified a bug in the model.
The flaw was rectified after its detection, strengthening the correctness of
the leader election protocol.
After this flaw was corrected, \texttt{romp} detected no more issues.

\subsection{The Trade-offs Between Exhaustive Enumeration and Parallel Random Walks}\label{subsec:the-trade-offs-between-exhaustive-enumeration-and-parallel-random-walks}

The comparison between \texttt{rumur} and \texttt{romp} highlights the
trade-offs associated with the choice between exhaustive enumeration and
parallel random walks in the context of model checking.

\texttt{rumur}, with its exhaustive enumeration approach,
guarantees that every possible state is explored, providing a definitive
answer regarding the presence of flaws.
This method is particularly suitable for smaller models with manageable state
spaces, where it is feasible to exhaustively cover all states.
When confronted with models characterized by vast state spaces, however,
such as the leader election protocol, the exhaustive enumeration approach
becomes impractical due to the rapid growth in the number of states to
be explored.

On the other hand, \texttt{romp}, leveraging parallel random walks,
offers a scalable and efficient approach to model checking.
By simultaneously exploring multiple states and performing random transitions,
\texttt{romp} covers an extensive state space in a short amount of time.
This approach is well-suited for models with large and complex state spaces,
where exhaustive enumeration is infeasible.
The probabilistic nature of parallel random walks introduces a level of
uncertainty, however, as it cannot provide strong guarantees of complete
bug-freedom in the model.

The choice between \texttt{rumur} and \texttt{romp} depends on the
characteristics of the model and the specific requirements of the analysis.
For smaller models where exhaustive coverage is possible, \texttt{rumur} can
provide definitive results.
Conversely, for models with enormous state spaces where exhaustive
enumeration is infeasible, \texttt{romp} offers a practical and scalable
approach to detect bugs.

\subsection{Areas for Future Research}\label{subsec:areas-for-future-research}

The exploration of scalable model checking using \texttt{rumur} and
\texttt{romp} has provided valuable insights into their effectiveness and
trade-offs.
However, there are several areas for future research that can further enhance
model-checking techniques.

One direction for future research is the development of hybrid approaches
that combine the strengths of exhaustive enumeration and parallel random walks.
Such approaches could intelligently switch between strategies based on the
characteristics of the model or employ heuristics to determine the most
suitable approach at different stages of the analysis.
A hybrid approach could even be used to make exhaustive guarantees for
granular functions and rules, and then parallel random walks could be used to
analyze the other functions and rules with too large a state space to entirely
enumerate.
By leveraging the benefits of both exhaustive enumeration and parallel random
walks, these hybrid techniques could improve the efficiency and accuracy of
model checking, especially for models with varying levels of complexity.

Another direction for future research is to minimize the state space through
induction.
The leader election protocol is a good example of how this could be effective:
if the protocol works for, say, three processes, four processes,
and five processes, then one can apply induction to show that the protocol will
work for any arbitrary number of processes beyond that.
Such an approach could detect this and exhaustively enumerate the states of
a model with three, four, and five processes, and then apply induction to prove
that the protocol will work for four hundred processes, or for even higher
magnitudes.

Finally, there is room for future research in regard to scalar sets;
the case studies of the tic-tac-toe strategy and the leader election protocol
use primitive C-style arrays to represent all parts of the state.
We have done no experimentation on models with scalar sets.
