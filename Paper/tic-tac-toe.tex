%%%%%%%%%%%%%%%%%%%%%%%%%%%%%%%%%%%%%%%%%%%%%%%%%%%%%%%%%%%%%%%%%%%%%%%%%%%%%%%%
%  File:   tic-tac-toe.tex                                                     %
%  Author: Cayden Lund (cayden.lund@utah.edu)                                  %
%          Student, University of Utah                                         %
%                                                                              %
%  Brief:  Contains the section on the tic-tac-toe case study.                 %
%%%%%%%%%%%%%%%%%%%%%%%%%%%%%%%%%%%%%%%%%%%%%%%%%%%%%%%%%%%%%%%%%%%%%%%%%%%%%%%%

\section{Case Study 1: A Tic-Tac-Toe Strategy}\label{sec:tic-tac-toe}

Tic-Tac-Toe is a classic two-player game played on a $3 \times 3$ grid.
The game is also known as \textit{noughts and crosses} or \textit{X's and O's},
with one player marking the squares with X's and the other with O's.
The objective of the game is for a player to get three marks in a row of that
player's type, either horizontally, vertically, or diagonally, before the
opponent.
If all the squares are filled without a winner, the game is considered a draw.

\subsection{The Tic-Tac-Toe Strategy}\label{subsec:the-tic-tac-toe-strategy}

The goal of this strategy is to never allow the opponent to win.
This means that either the agent employing the strategy
(from here on, \textit{the player}) wins the game, or the game ends in a draw.

In this strategy, the player is assumed to go first.
The player's mark is \textit{X}.
The opponent's mark is \textit{O}.

\begin{figure}
    \begin{center}
        \tictactoeboard{+}{ }{ }{ }{ }{ }{ }{ }{ }
    \end{center}
    \caption{The player's first move.}
    \label{fig:ttt-move-1}
\end{figure}

The player's first move is in the top-left corner of the board,
as shown in Figure~\ref{fig:ttt-move-1}.
From there, the opponent may place a mark in any of the eight remaining
locations on the board.
Depending on the opponent's next move, the player will react accordingly.

\begin{figure}
    \begin{center}
        \vspace{1em}

        \begin{tabular}{c c c}
            \tictactoeboard{X}{*}{ }{ }{+}{ }{ }{ }{ } &
            \tictactoeboard{X}{ }{*}{ }{ }{ }{ }{ }{+} &
            \tictactoeboard{X}{ }{ }{*}{+}{ }{ }{ }{ }
        \end{tabular}

        \vspace{2em}

        \begin{tabular}{c c c}
            \tictactoeboard{X}{ }{ }{ }{*}{ }{ }{ }{+} &
            \tictactoeboard{X}{ }{ }{ }{+}{*}{ }{ }{ } &
            \tictactoeboard{X}{ }{ }{ }{ }{ }{*}{ }{+}
        \end{tabular}

        \vspace{2em}

        \begin{tabular}{c c c}
            \tictactoeboard{X}{ }{ }{ }{+}{ }{ }{*}{ } &
            \tictactoeboard{X}{+}{ }{ }{ }{ }{ }{ }{*} &
            \phantom{\tictactoeboard{X}{ }{ }{ }{ }{ }{ }{ }{ }}
        \end{tabular}
    \end{center}
    \caption{The player's second move.}
    \label{fig:ttt-move-2}
\end{figure}

Figure~\ref{fig:ttt-move-2} shows the lookup table for the player's moves
for the second turn of the game.
For brevity, the entire strategy isn't described in the paper;
see the accompanying file \texttt{tictactoe\_buggy.m} for details.

\subsection{The Flaw in the Strategy}\label{subsec:the-flaw-in-the-strategy}

\begin{figure}
    \begin{center}
        \begin{tabular}{c c c}
            \tictactoeboard{+}{ }{ }{ }{ }{ }{ }{ }{ } &
            \tictactoeboard{X}{ }{ }{ }{ }{ }{ }{ }{*} &
            \tictactoeboard{X}{+}{ }{ }{ }{ }{ }{ }{O}
        \end{tabular}

        \vspace{2em}

        \begin{tabular}{c c c}
            \tictactoeboard{X}{X}{*}{ }{ }{ }{ }{ }{O} &
            \tictactoeboard{X}{X}{O}{ }{ }{+}{ }{ }{O} &
            \tictactoeboard{X}{X}{O}{ }{ }{X}{*}{ }{O}
        \end{tabular}

        \vspace{2em}

        \begin{tabular}{c c c}
            \tictactoeboard{X}{X}{O}{ }{+}{X}{O}{ }{O} &
            \tictactoeboard{X}{X}{O}{ }{X}{X}{O}{*}{O} &
            \phantom{\tictactoeboard{X}{ }{ }{ }{ }{ }{ }{ }{ }}
        \end{tabular}
    \end{center}
    \caption{The losing sequence.}
    \label{fig:ttt-losing-sequence}
\end{figure}

Under a certain sequence of moves, the tic-tac-toe strategy,
which is under examination in this case study,
inadvertently allows the opponent to secure a winning position.
Specifically, there exists a particular sequence of moves where the strategy
fails to anticipate a strategic counter-move from the opponent,
resulting in an unfavorable board configuration.
This sequence is shown in Figure~\ref{fig:ttt-losing-sequence}.
The flaw in the strategy opens up an opportunity for the opponent to
strategically place marks and create a winning pattern,
exploiting the strategy's vulnerability.
By identifying and addressing this specific sequence of moves,
we aim to rectify the flaw and enhance the strategy's effectiveness in
preventing the opponent from achieving a winning position.

\subsection{Identifying the Flaw with \texttt{rumur} and \texttt{romp}}\label{subsec:identifying-the-flaw-with-rumur-and-romp}

To conduct a thorough analysis of the tic-tac-toe strategy,
we encoded it as a Murphi model.
The process involved capturing the key elements and decision-making rules of
the strategy within the formal framework of the Murphi modeling language.
We defined the state space of the model, representing the various
configurations of the tic-tac-toe game board
(i.e., the positions occupied by X's and O's) and the current player's turn.
Additionally, we encoded the strategy's logic for determining optimal moves
based on the current game state.
By encapsulating the tic-tac-toe strategy within the Murphi model,
we created a formal representation that allowed us to apply model-checking
techniques, such as using \texttt{rumur} and \texttt{romp},
to systematically analyze and evaluate the strategy's effectiveness and
identify any underlying flaws or vulnerabilities.
This model can be found in the accompanying file \texttt{tictactoe\_buggy.m}.

First, we analyze the model with \texttt{rumur}.
\texttt{rumur} has a very user-friendly command-line interface:
we first invoke the \texttt{rumur} command and supply it with three arguments:
the filename of the model, \texttt{-o} (for ``out''), and the output filename
for a generated C file.
\texttt{rumur} then parses the model and generates a C file that will enumerate
the entire state space of the model and identify any underlying faults.
After that, we compile the C file and run the generated binary.

Running the compiled program reveals that there was a sequence in which
Player 2 wins the game.
Further, it shows the complete trace of the losing sequence from the start
state to the losing state, with all transitions.
This sequence is shown in Figure~\ref{fig:ttt-losing-sequence}.

Analysis of the model with \texttt{romp} is not quite as straightforward.
First, we generate C++ code for the model by invoking the \texttt{romp}
command with four arguments: \texttt{-S} (for ``simple traces''), the
filename of the model, \texttt{-o} for (``out''), and the output filename
for a generated C++ file.
\texttt{romp} then parses the model and generates a C++ program that will
perform a parallel random walk of the model's state space.
That is, several threads will simultaneously run the model from the starting
state and perform random transitions from the starting state.
This process is repeated several times, so an enormous state space is covered
in a short amount of time.
We compile the C++ program and run the generated binary with the
\texttt{-t} flag (for ``traces'').

Upon being run, the generated program reports that there is a losing sequence
in which the opponent wins the game.
Because of the random nature of these parallel walks, any readers repeating the
experiment here may need to run the binary a couple of times for it to come
across a losing sequence.
The program's invocation saves ninety-six different traces as JSON files,
and reports which one of the traces contains the losing sequence.
For every state in the sequence, this trace lists every rule and states whether
it applies to the state, before selecting and applying a rule.
Thus, this trace is very large and difficult to read by hand.
We recommend using some kind of filtering utility like \texttt{sed} to trim
down the trace file to a readable size, holding only relevant information.

\subsection{Correcting the Flaw}\label{subsec:correcting-the-flaw}

\begin{figure}
    \begin{center}
        \tictactoeboard{X}{X}{O}{ }{\#}{X}{*}{\#}{O}
    \end{center}
    \caption{The disadvantageous state. The opponent can win in two ways.}
    \label{fig:ttt-trouble-state}
\end{figure}

By analyzing the sequence, we see that this sequence of moves allows the
opponent to reach a state where there are two unblocked two-in-a-rows.
That is, after the opponent takes a turn, there are two spaces on the board
where, if the opponent were to take another turn, the opponent would win.
This guarantees that the opponent will win, as the player can only place a mark
in one of these two spaces.
This state is shown in Figure~\ref{fig:ttt-trouble-state},
with \tictactoeletter{\#} representing a blank space that the opponent can
fill to win the game.
The player can only fill one of these two spaces on the player's turn.

\begin{figure}
    \begin{center}
        \vspace{1em}

        \begin{tabular}{c c c}
            \tictactoeboard{+}{ }{ }{ }{ }{ }{ }{ }{ } &
            \tictactoeboard{X}{ }{ }{ }{ }{ }{ }{ }{*} &
            \tictactoeboard{X}{ }{+}{ }{ }{ }{ }{ }{O}
        \end{tabular}

        \vspace{2em}

        \begin{tabular}{c c c}
            \tictactoeboard{X}{*}{X}{ }{ }{ }{ }{ }{O} &
            \tictactoeboard{X}{O}{X}{ }{ }{ }{+}{ }{O} &
            \tictactoeboard{X}{O}{X}{*}{ }{ }{X}{ }{O}
        \end{tabular}

        \vspace{2em}

        \begin{tabular}{c c c}
            \tictactoeboard{X}{O}{X}{O}{+}{ }{X}{ }{O} &
            \phantom{\tictactoeboard{ }{ }{ }{ }{ }{ }{ }{ }{ }} &
            \phantom{\tictactoeboard{ }{ }{ }{ }{ }{ }{ }{ }{ }}
        \end{tabular}
    \end{center}
    \caption{A winning sequence with the new strategy.}
    \label{fig:ttt-win}
\end{figure}

To correct the flaw in the strategy, we need to prevent this from happening.
We change the second move of the game so that the player, instead of putting
a mark in the top-center space, will put a mark in the top-right space.
By making this change, the opponent is never able to reach the advantageous
state where there are two blank spaces that the opponent can fill to win the
game.
A sample game played with the fixed strategy is shown in
Figure~\ref{fig:ttt-win}.
This corrected strategy can be found in the accompanying file
\texttt{tictactoe\_fixed.m}.
