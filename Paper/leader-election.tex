%%%%%%%%%%%%%%%%%%%%%%%%%%%%%%%%%%%%%%%%%%%%%%%%%%%%%%%%%%%%%%%%%%%%%%%%%%%%%%%%
%  File:   leader-election.tex                                                 %
%  Author: Cayden Lund (cayden.lund@utah.edu)                                  %
%          Student, University of Utah                                         %
%                                                                              %
%  Brief:  Contains the section on the leader election protocol case study.    %
%%%%%%%%%%%%%%%%%%%%%%%%%%%%%%%%%%%%%%%%%%%%%%%%%%%%%%%%%%%%%%%%%%%%%%%%%%%%%%%%

\section{Case Study 2: The Leader Election Protocol}\label{sec:leader-election}

\subsection{The Leader Election Protocol}\label{subsec:the-leader-election-protocol}

In this case study, we examine the leader election protocol implemented in
the Murphi modeling language.
The leader election protocol aims to identify the process with the highest
value among a group of four hundred processes.
These processes are arranged in a circle, where each process can only receive
messages from its left and send messages to its right.
Each process initially starts in the \texttt{ACTIVE} state and is assigned a
unique value.
The protocol operates based on a set of rules and message passing between the
processes.

The leader election protocol defines two main states for the processes:
\texttt{ACTIVE} and \texttt{PASSIVE}.
Initially, all processes are in the \texttt{ACTIVE} state.
A process transitions to the \texttt{PASSIVE} state when it learns that its
value cannot be the maximum value among all processes.

In the \texttt{PASSIVE} state, a process only receives messages marked
\texttt{ONE} or \texttt{TWO} from its left and forwards the messages to its
right.
If it receives a message marked \texttt{FINAL}, then it records the value,
transitions to the \texttt{FINISHED} state, and then forwards the message
to the process on its right.

In the \texttt{ACTIVE} state, each process follows a set of rules for
exchanging messages and determining the maximum value.
First, each \texttt{ACTIVE} process sends its own value to the right and then
awaits the value of the closest \texttt{ACTIVE} process on its left.
This message, tagged as \texttt{ONE}, is saved for later use as a comparison
to determine the maximum value.
After saving the \texttt{ONE}-tagged message, the \texttt{ACTIVE} process then
forwards that message to the process on its right with the tag \texttt{TWO}.

If an \texttt{ACTIVE} process receives a value that is different from the one
it sent, it waits for the value of the second-closest \texttt{ACTIVE} process
on its left.
This message, tagged as \texttt{TWO}, is then compared with the process's value
and the previously-saved value of the closest \texttt{ACTIVE} process.
If the value of the current process is the largest of the three
(i.e., the current process's value, the value of the closest \texttt{ACTIVE}
process on the left, and the value of the second-closest \texttt{ACTIVE}
process on the left),
the current process retains its value and remains \texttt{ACTIVE}.
Otherwise, it transitions to the \texttt{PASSIVE} state.

When an \texttt{ACTIVE} process receives the same value it sent,
it concludes that it is the only \texttt{ACTIVE} process with the maximum value.
This is a safe conclusion because each process has a unique value, and because
a process receiving its own value means that the process is its own closest
active process, meaning that it is the only \texttt{ACTIVE} process.
When this happens, the process sends its value once more,
tagged as \texttt{FINAL}, and transitions to the \texttt{FINISHED} state.

If a process receives a message tagged as \texttt{FINAL}, it saves the
received value as the maximum value, forwards the message to its right,
and transitions to the \texttt{FINISHED} state.

When a process in the \texttt{FINISHED} state receives a message tagged
\texttt{FINAL}, then all processes should be in the \texttt{FINISHED} state.
At this point, the assertions are made
that all processes are in the \texttt{FINISHED} state;
that there are no pending, unreceived messages;
and that the saved maximum value is correct for all processes.

\subsection{The Flaw in the Model}\label{subsec:the-flaw-in-the-model}

We created a Murphi model of the leader election protocol.
This model has a rule named
``Active process: awaiting \texttt{ONE}: found max value''.
The condition upon which this rule can be applied is incorrect;
the model does not check whether the received value tagged \texttt{ONE} is
equivalent to the current process's value.
That is, any \texttt{ACTIVE} process receiving a value tagged \texttt{ONE}
can apply this rule, mark itself as \texttt{FINISHED}, and send a message
tagged \texttt{FINAL}, even if the received \texttt{ONE}-tagged value is not
equivalent to the process's actual value.

This model can be found in the accompanying file named
\texttt{leader\_election\_buggy.m}.

\subsection{Identifying the Flaw with \texttt{romp}}\label{subsec:identifying-the-flaw-in-romp}

The state space of this model is enormous.

Initially, we utilize the \texttt{romp} command to generate C++ code for the
model, using the same four arguments as we did for the tic-tac-toe strategy.
The \texttt{-S} flag is specified to generate ``simple traces'' of the model.
We also provide the filename of the model and use the \texttt{-o} flag to
indicate the desired output filename for the resulting C++ file.
Upon invocation, \texttt{romp} parses the model and generates a C++ program
that will perform a parallel random walk through the state space.
Multiple threads concurrently walk through the model from the initial state
making random transitions, repeating this process several times to
cover an extensive state space quickly.
After compiling the generated C++ program, we execute the resulting binary
with the \texttt{-t} flag, which instructs the program to record traces from
the model's execution.

The generated program upon execution detects the presence of several sequences
where the assertions are violated.
That is, either there are unreceived messages,
not all processes are in the \texttt{FINISHED} state,
or the processes haven't recorded the correct \texttt{FINAL} value.
The program's execution produces ninety-six distinct traces,
saved as JSON files, and identifies the traces that demonstrate violated
assertions.
These trace files are substantive, and tedious to interpret manually;
as such, we suggest using a filtering utility like \texttt{sed}
to reduce the trace file to a manageable size for easier analysis.

These trace files show that processes are incorrectly entering the
\texttt{FINISHED} state and sending a \texttt{FINAL} message.
We can see that the ``Active process: awaiting \texttt{ONE}: found max value''
rule is being applied when not appropriate.
Upon viewing the condition upon which this rule can be applied, we see that
the rule is not checking that the received value is equivalent to the
current process's actual value;
we tighten this condition and the resulting model
(\texttt{leader\_election\_fixed.m}) successfully passes all checks.

\subsection{A Word on \texttt{rumur}}\label{subsec:a-word-on-rumur}

The leader election protocol involves a considerable state space due to the
presence of four hundred processes and various possible combinations of their
states.
Unfortunately, the size of this state space surpasses the capabilities of
\texttt{rumur}, rendering it ineffective for comprehensive analysis.
\texttt{rumur} relies on explicit enumeration, systematically exploring every
potential state of a model.
However, the enormous scale of the state space in the leader election
protocol makes it impractical to exhaustively cover all possible states using
\texttt{rumur}.
Consequently, we are unable to obtain meaningful results from \texttt{rumur}
regarding the correctness of the model or the identification of flaws.
To overcome this limitation, we employ the parallel random walk strategy
offered by \texttt{romp}, which allows us to navigate the extensive state
space efficiently and detect potential bugs in the leader election protocol.
