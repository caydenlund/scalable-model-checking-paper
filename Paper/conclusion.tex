%%%%%%%%%%%%%%%%%%%%%%%%%%%%%%%%%%%%%%%%%%%%%%%%%%%%%%%%%%%%%%%%%%%%%%%%%%%%%%%%
%  File:   conclusion.tex                                                      %
%  Author: Cayden Lund (cayden.lund@utah.edu)                                  %
%          Student, University of Utah                                         %
%                                                                              %
%  Brief:  Contains the concluding section.                                    %
%%%%%%%%%%%%%%%%%%%%%%%%%%%%%%%%%%%%%%%%%%%%%%%%%%%%%%%%%%%%%%%%%%%%%%%%%%%%%%%%

\subsection{Summary of the Main Findings}\label{subsec:summary-of-the-main-findings}

In this paper, we conducted a practical exploration of scalable model
checking using the \texttt{rumur} and \texttt{romp} tools for the Murphi
modeling language.
We presented two case studies: a tic-tac-toe strategy and a model for the
leader election protocol.
Through these case studies, we evaluated the performance and effectiveness of
\texttt{rumur} and \texttt{romp} in identifying and correcting flaws in the
models.

In the tic-tac-toe case study, we encoded the strategy as a Murphi model and
used both \texttt{rumur} and \texttt{romp} to identify a flaw that allowed
the opponent to win under a certain sequence of moves.
After correcting the flaw, both tools confirmed the model's correctness.
This demonstrated the ability of both \texttt{rumur} and \texttt{romp} to
effectively detect and rectify flaws in smaller models.

In the leader election protocol case study,
we analyzed a model with a large state space,
making it infeasible for \texttt{rumur} to perform exhaustive enumeration.
\texttt{romp} successfully detected a bug in the model,
highlighting its scalability and efficiency for models with enormous
state spaces.
The bug was corrected, ensuring the model's adherence to the desired
specifications.

\subsection{Final Remarks}\label{subsec:final-remarks}

Our exploration of scalable model checking using \texttt{rumur} and
\texttt{romp} has provided valuable insights into their performance,
effectiveness, and trade-offs.
The choice between using an exhaustive state-space enumeration and
a parallel random walk depends on the characteristics of the model and the
scale of the state space.
\texttt{rumur} excels in smaller models, offering exhaustive enumeration and
definitive results, while \texttt{romp} is more suitable for larger models,
providing a scalable approach with efficient coverage of enormous state spaces.

Model checking plays a crucial role in ensuring the reliability, correctness,
and safety of complex systems.
By systematically exploring the state space and analyzing system behavior
against specified properties, model checking aids in identifying critical
flaws and vulnerabilities.
The practical examination of scalable model checking techniques using
\texttt{rumur} and \texttt{romp} contributes to the advancement of
formal verification methods and provides engineers and developers with
powerful tools to enhance system dependability.

The exploration of scalable model checking using \texttt{rumur} and
\texttt{romp} has provided valuable insights into their effectiveness and
trade-offs.
Future research directions include the development of hybrid approaches that
combine exhaustive enumeration and parallel random walks to leverage the
strengths of both strategies, enhancing the efficiency and accuracy of model
checking for models with varying levels of complexity.
Additionally, research can focus on minimizing the state space through
induction, allowing for the application of proven protocols to larger-scale
systems.
Furthermore, investigating the use of scalar sets in models presents an area
for future research, as the current case studies have primarily used
primitive C-style arrays.
These research avenues will further advance model-checking techniques and
expand their applicability in various domains.

In conclusion, scalable model checking techniques, exemplified by
\texttt{rumur} and \texttt{romp}, offer valuable insights and capabilities
for system verification.
By leveraging these tools and exploring future research directions,
we can ensure the correctness and reliability of increasingly complex systems,
mitigating risks and enabling the development of robust and dependable
software and hardware solutions.
